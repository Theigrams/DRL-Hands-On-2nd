% C-c C-o to insert the block

% Individual equation: equation* block
% Inline equation \begin{math}\frac{sin(x)}{x}\end{math}
\documentclass{article}

\usepackage{amsmath,amssymb}

\ifdefined\ispreview
\usepackage[active,tightpage]{preview}
\PreviewEnvironment{math}
\PreviewEnvironment{equation*}
\fi

\DeclareMathOperator{\E}{\mathbb{E}}
\DeclareMathOperator*{\argmin}{arg\,min}
\DeclareMathOperator*{\argmax}{arg\,max}

\begin{document}

Page 1, Values and policy

Formally it could be written as \begin{math}\pi(s) = \argmax_a Q(s, a)\end{math}, which means that the
result of our policy \begin{math}\pi\end{math} at every state s is the action with the largest Q.

In math notation, policy is usually represented as \begin{math}\pi(s)\end{math}, we’ll use this notation as well.

Page 4, Policy gradients

We define policy gradient as \begin{math}\nabla J \approx \E[Q(s, a) \nabla\log \pi(a|s)]\end{math}. Of course, there is a
strong proof of this, but it’s not that important. Which is much more important
is the semantic of this expression.

From the practical point of view, policy gradient methods could be implemented
as performing optimisation of this loss function: \begin{math}\mathcal{L} = -Q(s, a) \log \pi(a|s)\end{math}.


Page 5, REINFORCE

\begin{enumerate}
\item Initialize the network with random weight.
\item Play N full episodes, saving their \begin{math}(s_i, a_i, r_i, s’_i)\end{math} transitions.
\item
  For every step t of every episode k calculate discounted total reward for
  subsequent steps \begin{math}Q_{k,t} = \sum_{i=0}\gamma^ir_i\end{math}
\item Calculate loss function for all transitions \begin{math}\mathcal{L} =
  -\sum_{k,t}Q_{k,t}\log(\pi(s_{k,t}, a_{k,t}))\end{math}
\item Perform SGD update of weights minimizing the loss.
\item Repeat from step 2 until converged.
\end{enumerate}


Page 6, CartPole Example

To do this efficiently, we calculate the reward from the end of the local reward list. Indeed, the last step of the episode will have the total reward equal to its local reward. The step before the last will have the total reward of
\begin{math}r_{t-1} + \gamma r_t\end{math} (if t is an index of the last step).


Page 12, Full episodes are required

When we talked about DQN we’ve seen that in practice it’s fine to replace the
exact value for discounted reward with our estimation using 1-step Bellman
equation \begin{math}Q(s, a) = r_a + \gamma V(s')\end{math}.


Page 13, High gradient variance

In policy gradients formula \begin{math}\nabla J \approx \E[Q(s, a) \nabla\log \pi(a|s)]\end{math} we have gradient
proportional to the discounter reward from the given state.

Page 13, Exploration

In math notation, entropy of the policy is defined as: \begin{math}H(\pi) =
  -\sum \pi(a|s)\log \pi(a|s)\end{math}

Page 18, Results

The entropy is decreasing over time from 0.69 to 0.52. The starting value corresponds to the maximum entropy with two actions, which is approximately 0.69:
\begin{equation*}
H(\pi) = -\sum_a \pi(a|s)\log \pi(a|s) = - (\frac{1}{2} \log(\frac{1}{2}) + \frac{1}{2} \log(\frac{1}{2})) \approx 0.69
\end{equation*}

The fact that the entropy is decreasing during the training, as indicated by the following chart,   shows that our policy is moving from the uniform distribution to more deterministic actions.

\end{document}
