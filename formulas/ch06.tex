% C-c C-o to insert the block

% Individual equation: equation* block
% Inline equation \begin{math}\frac{sin(x)}{x}\end{math}
\documentclass{article}

\usepackage{amsmath,amssymb}

\ifdefined\ispreview
\usepackage[active,tightpage]{preview}
\PreviewEnvironment{math}
\PreviewEnvironment{equation*}
\fi

\DeclareMathOperator{\E}{\mathbb{E}}
\DeclareMathOperator*{\argmin}{arg\,min}
\DeclareMathOperator*{\argmax}{arg\,max}

\begin{document}

Page 3, Tabular Q-Learning

\begin{enumerate}
  \item Start with empty table mapping states to values of actions,
  \item
    By interacting with the environment, obtain tuple (s, a, r, s').
  \item
    Update Q(s, a) value using Bellman approximation: \begin{math}Q(s,a)
      \leftarrow r + \gamma \max\limits_{a' \in A}Q(s',a')\end{math}
  \item
    Repeat step 2.
\end{enumerate}


...using learning rate $\alpha$ with value from 0 to 1:

\begin{equation*}
  Q(s,a) \leftarrow (1 - \alpha)Q(s,a) + \alpha (r + \gamma \max\limits_{a' \in A}Q(s',a'))
\end{equation*}

The final version of the algorithm is here:

\begin{enumerate}
  \item Start with empty table for Q(s, a).
  \item Obtain (s, a, r, s') from the environment.
  \item Make Bellman update: \begin{math}Q(s, a) \leftarrow (1 - \alpha) Q(s, a) + \alpha (r + \gamma \max\limits_{a' \in A} Q(s',a'))\end{math}
  \item Check convergence conditions, if not met, repeat from step 2.
\end{enumerate}


Page 8, Deep Q-learning

With this in mind, let's make modifications to the Q-learning algorithm:
\begin{enumerate}
  \item Initialize Q(s, a) with some initial approximation,
  \item By interacting with the environment, obtain tuple (s, a, r, s').
  \item Calculate loss:
    \begin{math}\mathcal{L} = (Q(s,a) - r)^2\end{math} if episode has ended or
      \begin{math}\mathcal{L} = (Q(s,a) - (r + \gamma \max\limits_{a' \in
          A}Q_{s',a'}))^2\end{math} otherwise.
  \item Update Q(s, a) using SGD algorithm by minimizing the loss in respect to model parameters.
  \item Repeat step 2 until converged.
\end{enumerate}

Page 11, Final form of DQN training
\begin{enumerate}
  \item Initialize parameters for Q(s, a) and \begin{math}\hat{Q}(s, a)\end{math} with random
    weights, \begin{math}\epsilon \leftarrow 1.0\end{math}, and empty replay buffer
  \item With probability \begin{math}\epsilon\end{math} select a random action a, otherwise
    \begin{math}a = \argmax_aQ(s, a)\end{math}
  \item Execute action a in emulator and observe reward r and next state s'.
  \item Store transition (s, a, r, s') in the replay buffer.
  \item Sample random minibatch of transitions from replay buffer.
  \item For every transition in the buffer calculate target \begin{math}y = r\end{math}
    if episode has ended at this step or
    \begin{math}y = r + \gamma \max\limits_{a' \in A} \hat{Q}(s',a')\end{math}
  \item Calculate loss: \begin{math}\mathcal{L} = (Q(s, a) - y)^2\end{math}
  \item Update Q(s, a) using SGD algorithm by minimizing the loss in respect to model parameters.
  \item Every N steps copy weights from Q to \begin{math}\hat{Q}\end{math}
  \item Repeat step 2 until converged.
\end{enumerate}


Page 24

As a reminder, there is the loss expression we need to calculate: \begin{math}\mathcal{L} = (Q(s, a) -
(r + \gamma \max\limits_{a' \in A} \hat{Q}(s',a')))^2\end{math} for steps which wasn`t
  at the end of the episode or
  \begin{math}\mathcal{L} = (Q(s, a) - r)^2\end{math} for final steps.
\end{document}
